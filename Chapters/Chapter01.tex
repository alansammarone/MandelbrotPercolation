% Chapter 1

\chapter{Classic percolation}

\label{ch:classicpercolation} % For referencing the chapter elsewhere, use \autoref{ch:introduction} 

%----------------------------------------------------------------------------------------

%----------------------------------------------------------------------------------------

\section{Basic definitions}

We begin with a graph G representing a slattice of size $L$ in $d$ dimensions, such that $|G| = L^d$. 
Each node in the graph is independently set to be "alive"/"white" with probability $p$ and "dead"/"black" with probability $1-p$. $p$ is often called the occupation probability.
Our goal is to study the properties of the lattice after this "coloring" has taken place. This is what such a lattice looks like: (IMAGE)


\begin{defn}
A cluster is a set of white connected nodes in the graph.
\end{defn}


\begin{defn}
A lattice is said to have percolated if there exists a macroscopic cluster, i.e. a cluster which spans the whole lattice.
\end{defn}

For the case of $d<\inf$, one can (arbitrarily) pick a dimension $i$ from ${1, 2, ... d}$ and use it as the defining dimension for percolation, i.e. a lattice has percolated if there exists a cluster that intersects both boundaries of the lattice. For $d=2$, for example, we can use the convention that a lattice has percolated if there is a cluster that connects the top boundary and the bottom boundary (left-right would be equally good).

The process is, of course, random. So we define an indicator random variable $H_{p,L}$ which represents whether a particular lattice has percolated: 

% \[
% 	
%     \begin{dcases}
%         
%         
%     \end{dcases}
% \]
\begin{defn}
\[
	H_{p,L} = 
    \begin{dcases}
        1 & \text{if lattice has percolated} \\
        0 & \text{otherwise} \\
    \end{dcases}
\]
\end{defn}
As we shall see, percolation models are the simplest models that exibit a phase transition, meaning that there exists a particular occupation probability $p_c$ at which the behavior of the system is expected to change dramatically. This change will be reflected in a number of quantities of interest which will study in the next sections. It turns out that $p_c$ of the probability of seeing a percolating cluster.
As we vary $p$, larger and larger clusters are expected to form. However, in the case that $L < \inf$ and $p < 1$, there is always a possilibity that no percolating cluster will be observed (i.e even if $p = 0.999$ there is a non-zero probability all nodes are dead). For this reason, $p_c$ is not well defined for finite $L$, and we'll  define it in the limit $L \xrightarrow \inf$. 

We can think of $E[H_{p,L}]$ as the probability that a particular lattice will percolate. As we increase $L$, the behaviour of $E[H_{p,L}]$ as a function of $p$ approach a step function at $p=p_c$. Which means that in the limit of very big L, the lattice percolates with probability 1 for $p > p_c$, and does not percolate with probability 1 for $p < p_c$.  

\begin{defn}
The percolation threshold $p_c$ is the smallest occupation probability $p$ such that a percolating cluster exists with probability 1. 
\end{defn}



\section{1D Case: toy model}\label{sec:1d}

\subsection{Percolation threshold $p_c$}

\subsection{Critical exponent $\sigma$}

\subsection{Critical exponent $\gamma$}

\subsection{Critical exponent $\nu$}


%----------------------------------------------------------------------------------------

\section{2D case in square lattice}\label{sec:2dsquare}


\subsection{Critical exponents}

\subsection{Correlation length}

\subsection{Phase transitions}

\subsection{Real space renormalisation}


%----------------------------------------------------------------------------------------

\section{Bethe lattice}\label{sec:2dgeneral}




